\documentclass[9pt]{article}

\usepackage[a4paper]{geometry}
\usepackage[english]{babel}
\usepackage{longtable}
\usepackage{booktabs}

\usepackage[scaled]{helvet}
\renewcommand*\familydefault{\sfdefault}

\begin{document}

%start
%latex.default(as.matrix(tex.out[[i]]), file = file, append = TRUE,     caption = paste("Partion summary for cluster ", i, " consisting out of ",         table(partitioning(obj))[i], " plots.", ifelse(length(caption.text) >             0, paste(" ", caption.text, ".", sep = ""), ""),         sep = ""), rowname = NULL, booktabs = TRUE, longtable = TRUE,     lines.page = nrow(tex.out[[i]]), here = TRUE)%
\setlongtables\begin{longtable}{lllllllllllllll}\caption{Partion summary for cluster 1 consisting out of 28 plots.} \tabularnewline
\toprule
\multicolumn{1}{c}{taxon}&\multicolumn{1}{c}{layer}&\multicolumn{1}{c}{typical}&\multicolumn{1}{c}{stat}&\multicolumn{1}{c}{cons}&\multicolumn{1}{c}{cont}&\multicolumn{1}{c}{occu}&\multicolumn{1}{c}{out}&\multicolumn{1}{c}{spread}&\multicolumn{1}{c}{q0}&\multicolumn{1}{c}{q0.25}&\multicolumn{1}{c}{q0.5}&\multicolumn{1}{c}{q0.75}&\multicolumn{1}{c}{q1}&\multicolumn{1}{c}{summary}\tabularnewline
\midrule
\endfirsthead\caption[]{\em (continued)} \tabularnewline
\midrule
\multicolumn{1}{c}{taxon}&\multicolumn{1}{c}{layer}&\multicolumn{1}{c}{typical}&\multicolumn{1}{c}{stat}&\multicolumn{1}{c}{cons}&\multicolumn{1}{c}{cont}&\multicolumn{1}{c}{occu}&\multicolumn{1}{c}{out}&\multicolumn{1}{c}{spread}&\multicolumn{1}{c}{q0}&\multicolumn{1}{c}{q0.25}&\multicolumn{1}{c}{q0.5}&\multicolumn{1}{c}{q0.75}&\multicolumn{1}{c}{q1}&\multicolumn{1}{c}{summary}\tabularnewline
\midrule
\endhead
\midrule
\endfoot
\label{as.matrix}
Cerastium holosteoides&hl&&0.362&75&21&119& 98&3&0&0.225&0.30& 0.300& 0.7&IV (0/0.3/0.7, n = 21)\tabularnewline
\textbf{ Trisetum flavescens }&hl&&0.336&96&27&142&115&3&0&8.000&8.00&18.000&38.0&V (0/8/38, n = 27)\tabularnewline
Arrhenatherum elatius&hl&yes&0.298&50&14& 43& 29&3&0&0.000&0.35& 8.000&18.0&III (0/0.35/18, n = 14)\tabularnewline
Avenula pubescens&hl&yes&0.288&82&23& 78& 55&3&0&0.700&8.00&18.000&38.0&V (0/8/38, n = 23)\tabularnewline
Dactylis glomerata&hl&&0.260&93&26&158&132&3&0&3.000&4.00& 8.000&18.0&V (0/4/18, n = 26)\tabularnewline
Trifolium pratense&hl&&0.251&86&24&139&115&3&0&0.600&0.70& 4.000& 8.0&V (0/0.7/8, n = 24)\tabularnewline
Tragopogon orientalis&hl&&0.235&50&14& 69& 55&3&0&0.000&0.15& 0.300& 0.7&III (0/0.15/0.7, n = 14)\tabularnewline
Plantago lanceolata&hl&yes&0.216&86&24&116& 92&3&0&0.300&0.70& 0.700& 4.0&V (0/0.7/4, n = 24)\tabularnewline
Rhinanthus alectorolophus&hl&&0.211&64&18&101& 83&3&0&0.000&0.30& 4.000& 8.0&IV (0/0.3/8, n = 18)\tabularnewline
Lathyrus pratensis&hl&&0.168&25& 7& 66& 59&3&0&0.000&0.00& 0.075& 0.7&II (0/0/0.7, n = 7)\tabularnewline
Poa pratensis&hl&&0.167&50&14& 38& 24&3&0&0.000&0.35& 3.250& 4.0&III (0/0.35/4, n = 14)\tabularnewline
Ranunculus acris&hl&&0.162&82&23&155&132&3&0&0.700&0.70& 1.275& 4.0&V (0/0.7/4, n = 23)\tabularnewline
Crepis biennis&hl&&0.159&57&16& 80& 64&3&0&0.000&0.30& 1.275& 8.0&III (0/0.3/8, n = 16)\tabularnewline
Campanula patula&hl&&0.158&32& 9& 61& 52&3&0&0.000&0.00& 0.300& 0.7&II (0/0/0.7, n = 9)\tabularnewline
Alchemilla vulgaris agg.&hl&&0.142&57&16&124&108&3&0&0.000&0.30& 1.275& 4.0&III (0/0.3/4, n = 16)\tabularnewline
Leontodon hispidus&hl&&0.135&64&18& 96& 78&3&0&0.000&0.50& 4.000&18.0&IV (0/0.5/18, n = 18)\tabularnewline
Stellaria graminea&hl&&0.133&21& 6& 58& 52&3&0&0.000&0.00& 0.000& 0.7&II (0/0/0.7, n = 6)\tabularnewline
Ranunculus bulbosus&hl&yes&0.122&14& 4&  8&  4&3&0&0.000&0.00& 0.000& 0.7&I (0/0/0.7, n = 4)\tabularnewline
Vicia cracca&hl&&0.119&68&19& 92& 73&3&0&0.000&0.30& 0.300& 3.0&IV (0/0.3/3, n = 19)\tabularnewline
Arenaria serpyllifolia s.str.&hl&yes&0.112&21& 6&  7&  1&2&0&0.000&0.00& 0.000& 0.7&II (0/0/0.7, n = 6)\tabularnewline
Dactylorhiza maculata s.lat. (inkl. fuchsii)&hl&&0.107&11& 3& 30& 27&2&0&0.000&0.00& 0.000& 0.3&I (0/0/0.3, n = 3)\tabularnewline
Primula elatior&hl&&0.107&11& 3& 18& 15&2&0&0.000&0.00& 0.000& 0.3&I (0/0/0.3, n = 3)\tabularnewline
Cirsium oleraceum&hl&&0.100&46&13& 70& 57&3&0&0.000&0.00& 0.700& 4.0&III (0/0/4, n = 13)\tabularnewline
Aegopodium podagraria&hl&&0.092&43&12& 66& 54&3&0&0.000&0.00& 0.700& 8.0&III (0/0/8, n = 12)\tabularnewline
Colchicum autumnale&hl&&0.091&75&21& 98& 77&3&0&0.225&0.30& 0.700& 3.0&IV (0/0.3/3, n = 21)\tabularnewline
Bromus hordeaceus&hl&&0.087&11& 3&  4&  1&2&0&0.000&0.00& 0.000& 0.7&I (0/0/0.7, n = 3)\tabularnewline
Veronica chamaedrys&hl&&0.086&75&21&102& 81&3&0&0.225&0.30& 0.300& 8.0&IV (0/0.3/8, n = 21)\tabularnewline
Medicago lupulina&hl&&0.083&32& 9& 25& 16&3&0&0.000&0.00& 0.300& 3.0&II (0/0/3, n = 9)\tabularnewline
Veronica arvensis&hl&&0.082&14& 4& 16& 12&3&0&0.000&0.00& 0.000& 0.7&I (0/0/0.7, n = 4)\tabularnewline
Myosotis arvensis&hl&&0.082&14& 4& 11&  7&3&0&0.000&0.00& 0.000& 0.7&I (0/0/0.7, n = 4)\tabularnewline
Campanula scheuchzeri&hl&&0.082&14& 4& 11&  7&2&0&0.000&0.00& 0.000& 0.7&I (0/0/0.7, n = 4)\tabularnewline
Festuca pratensis s.str.&hl&&0.081&61&17&111& 94&3&0&0.000&1.85& 4.000&18.0&IV (0/1.85/18, n = 17)\tabularnewline
Prunella vulgaris&hl&&0.079&25& 7& 70& 63&3&0&0.000&0.00& 0.075& 4.0&II (0/0/4, n = 7)\tabularnewline
Salvia verticillata&hl&&0.075&14& 4& 10&  6&2&0&0.000&0.00& 0.000& 4.0&I (0/0/4, n = 4)\tabularnewline
Platanthera bifolia&hl&&0.071& 7& 2& 17& 15&2&0&0.000&0.00& 0.000& 0.3&I (0/0/0.3, n = 2)\tabularnewline
Lilium bulbiferum&hl&&0.071& 7& 2&  7&  5&3&0&0.000&0.00& 0.000& 0.3&I (0/0/0.3, n = 2)\tabularnewline
Crepis mollis&hl&&0.071& 7& 2&  6&  4&2&0&0.000&0.00& 0.000& 0.7&I (0/0/0.7, n = 2)\tabularnewline
Carduus defloratus s.lat.&hl&&0.071& 7& 2&  3&  1&2&0&0.000&0.00& 0.000& 0.3&I (0/0/0.3, n = 2)\tabularnewline
Knautia arvensis&hl&&0.069&29& 8& 23& 15&2&0&0.000&0.00& 0.300& 8.0&II (0/0/8, n = 8)\tabularnewline
Cruciata laevipes&hl&&0.067&36&10& 32& 22&3&0&0.000&0.00& 0.300& 4.0&II (0/0/4, n = 10)\tabularnewline
Gymnadenia conopsea&hl&&0.066&11& 3&  7&  4&2&0&0.000&0.00& 0.000& 0.7&I (0/0/0.7, n = 3)\tabularnewline
Heracleum sphondylium&hl&&0.066&32& 9& 54& 45&3&0&0.000&0.00& 0.300& 3.0&II (0/0/3, n = 9)\tabularnewline
Poa trivialis&hl&&0.066&43&12& 75& 63&3&0&0.000&0.00& 1.275& 8.0&III (0/0/8, n = 12)\tabularnewline
Taraxacum officinale agg.&hl&&0.062&36&10& 74& 64&3&0&0.000&0.00& 0.300& 3.0&II (0/0/3, n = 10)\tabularnewline
Lychnis flos-cuculi&hl&&0.061&14& 4& 31& 27&2&0&0.000&0.00& 0.000& 0.3&I (0/0/0.3, n = 4)\tabularnewline
Myosotis sylvatica&hl&&0.061&14& 4& 19& 15&3&0&0.000&0.00& 0.000& 0.3&I (0/0/0.3, n = 4)\tabularnewline
Holcus lanatus&hl&&0.061&14& 4& 10&  6&3&0&0.000&0.00& 0.000&18.0&I (0/0/18, n = 4)\tabularnewline
Hypericum maculatum s.str.&hl&&0.061&46&13& 97& 84&3&0&0.000&0.00& 0.300&18.0&III (0/0/18, n = 13)\tabularnewline
Galium album s.str.&hl&&0.059&46&13& 51& 38&3&0&0.000&0.00& 0.400& 4.0&III (0/0/4, n = 13)\tabularnewline
Pimpinella major&hl&&0.058&39&11& 65& 54&3&0&0.000&0.00& 0.700& 4.0&II (0/0/4, n = 11)\tabularnewline
Trifolium repens&hl&&0.056&54&15& 93& 78&3&0&0.000&0.30& 0.700& 4.0&III (0/0.3/4, n = 15)\tabularnewline
Ajuga reptans&hl&&0.055&18& 5& 26& 21&2&0&0.000&0.00& 0.000& 3.0&I (0/0/3, n = 5)\tabularnewline
Rhinanthus minor&hl&&0.055&25& 7& 45& 38&3&0&0.000&0.00& 0.075& 4.0&II (0/0/4, n = 7)\tabularnewline
Origanum vulgare&hl&&0.054& 7& 2&  5&  3&2&0&0.000&0.00& 0.000& 8.0&I (0/0/8, n = 2)\tabularnewline
Arabis hirsuta&hl&&0.054&21& 6&  9&  3&2&0&0.000&0.00& 0.000& 3.0&II (0/0/3, n = 6)\tabularnewline
Persicaria bistorta&hl&&0.052&18& 5& 36& 31&3&0&0.000&0.00& 0.000& 4.0&I (0/0/4, n = 5)\tabularnewline
Rumex acetosa&hl&&0.052&57&16&115& 99&3&0&0.000&0.30& 0.700& 4.0&III (0/0.3/4, n = 16)\tabularnewline
Euphorbia esula&hl&&0.051& 7& 2&  2&  0&1&0&0.000&0.00& 0.000& 0.7&I (0/0/0.7, n = 2)\tabularnewline
Viola tricolor&hl&&0.051& 7& 2&  2&  0&1&0&0.000&0.00& 0.000& 0.7&I (0/0/0.7, n = 2)\tabularnewline
Bromus erectus&hl&&0.049&11& 3&  9&  6&3&0&0.000&0.00& 0.000& 4.0&I (0/0/4, n = 3)\tabularnewline
Vicia sepium&hl&&0.049&21& 6& 50& 44&3&0&0.000&0.00& 0.000& 4.0&II (0/0/4, n = 6)\tabularnewline
Buphthalmum salicifolium&hl&&0.042& 7& 2&  9&  7&2&0&0.000&0.00& 0.000& 4.0&I (0/0/4, n = 2)\tabularnewline
Geum rivale&hl&&0.041&11& 3& 13& 10&2&0&0.000&0.00& 0.000& 4.0&I (0/0/4, n = 3)\tabularnewline
Anthoxanthum odoratum&hl&&0.039&39&11&106& 95&3&0&0.000&0.00& 0.700& 8.0&II (0/0/8, n = 11)\tabularnewline
Equisetum arvense&hl&&0.038& 7& 2& 15& 13&2&0&0.000&0.00& 0.000& 4.0&I (0/0/4, n = 2)\tabularnewline
Filipendula ulmaria&hl&&0.036&11& 3& 28& 25&2&0&0.000&0.00& 0.000& 3.0&I (0/0/3, n = 3)\tabularnewline
Cynosurus cristatus&hl&&0.036&32& 9& 96& 87&3&0&0.000&0.00& 3.000& 8.0&II (0/0/8, n = 9)\tabularnewline
Glechoma hederacea&hl&&0.031& 7& 2& 12& 10&3&0&0.000&0.00& 0.000& 0.3&I (0/0/0.3, n = 2)\tabularnewline
Stachys alpina&hl&&0.031& 7& 2&  8&  6&3&0&0.000&0.00& 0.000& 0.3&I (0/0/0.3, n = 2)\tabularnewline
Silene dioica&hl&&0.031& 7& 2&  8&  6&2&0&0.000&0.00& 0.000& 0.3&I (0/0/0.3, n = 2)\tabularnewline
Euphorbia cyparissias&hl&&0.031& 7& 2&  3&  1&2&0&0.000&0.00& 0.000& 0.3&I (0/0/0.3, n = 2)\tabularnewline
Leucanthemum ircutianum&hl&&0.031&32& 9& 75& 66&3&0&0.000&0.00& 0.300& 0.7&II (0/0/0.7, n = 9)\tabularnewline
Phleum pratense&hl&&0.029&21& 6& 37& 31&3&0&0.000&0.00& 0.000& 3.0&II (0/0/3, n = 6)\tabularnewline
Centaurea jacea&hl&&0.029&36&10& 95& 85&3&0&0.000&0.00& 0.300& 4.0&II (0/0/4, n = 10)\tabularnewline
Briza media&hl&&0.029&39&11& 68& 57&2&0&0.000&0.00& 0.700&18.0&II (0/0/18, n = 11)\tabularnewline
Festuca nigrescens&hl&&0.022&32& 9&108& 99&3&0&0.000&0.00& 0.700& 8.0&II (0/0/8, n = 9)\tabularnewline
Carex sylvatica&hl&&0.021&14& 4& 72& 68&3&0&0.000&0.00& 0.000& 0.7&I (0/0/0.7, n = 4)\tabularnewline
Centaurea scabiosa&hl&&0.021&18& 5& 14&  9&2&0&0.000&0.00& 0.000& 3.0&I (0/0/3, n = 5)\tabularnewline
Achillea millefolium agg.&hl&&0.019&14& 4& 69& 65&3&0&0.000&0.00& 0.000& 0.7&I (0/0/0.7, n = 4)\tabularnewline
Carex pallescens&hl&&0.018&14& 4& 73& 69&3&0&0.000&0.00& 0.000& 0.7&I (0/0/0.7, n = 4)\tabularnewline
Lotus corniculatus&hl&&0.017&36&10& 71& 61&3&0&0.000&0.00& 0.300& 0.7&II (0/0/0.7, n = 10)\tabularnewline
Bellis perennis&hl&&0.016&21& 6& 24& 18&3&0&0.000&0.00& 0.000& 0.3&II (0/0/0.3, n = 6)\tabularnewline
Chaerophyllum aureum&hl&&0.016&32& 9& 55& 46&3&0&0.000&0.00& 0.400& 4.0&II (0/0/4, n = 9)\tabularnewline
Agrostis capillaris&hl&&0.013&11& 3& 61& 58&3&0&0.000&0.00& 0.000& 3.0&I (0/0/3, n = 3)\tabularnewline
Narcissus radiiflorus&hl&&0.011&11& 3& 23& 20&2&0&0.000&0.00& 0.000& 4.0&I (0/0/4, n = 3)\tabularnewline
Cirsium arvense&hl&&0.011&14& 4& 12&  8&3&0&0.000&0.00& 0.000& 0.3&I (0/0/0.3, n = 4)\tabularnewline
Leucanthemum vulgare&hl&&0.009& 7& 2& 19& 17&2&0&0.000&0.00& 0.000& 0.7&I (0/0/0.7, n = 2)\tabularnewline
Rhinanthus glacialis&hl&&0.009& 7& 2& 16& 14&2&0&0.000&0.00& 0.000& 0.7&I (0/0/0.7, n = 2)\tabularnewline
Geranium sylvaticum&hl&&0.009&14& 4& 35& 31&2&0&0.000&0.00& 0.000& 0.7&I (0/0/0.7, n = 4)\tabularnewline
Carum carvi&hl&&0.009&25& 7& 58& 51&3&0&0.000&0.00& 0.075& 0.3&II (0/0/0.3, n = 7)\tabularnewline
Potentilla erecta&hl&&0.007& 7& 2& 51& 49&2&0&0.000&0.00& 0.000& 0.3&I (0/0/0.3, n = 2)\tabularnewline
Trollius europaeus&hl&&0.007&14& 4& 48& 44&3&0&0.000&0.00& 0.000& 0.7&I (0/0/0.7, n = 4)\tabularnewline
Ranunculus repens&hl&&0.006&11& 3& 38& 35&3&0&0.000&0.00& 0.000& 0.7&I (0/0/0.7, n = 3)\tabularnewline
Luzula multiflora s.str.&hl&&0.005& 7& 2& 31& 29&2&0&0.000&0.00& 0.000& 0.3&I (0/0/0.3, n = 2)\tabularnewline
Chaerophyllum hirsutum&hl&&0.005&18& 5& 64& 59&3&0&0.000&0.00& 0.000& 0.7&I (0/0/0.7, n = 5)\tabularnewline
Geranium phaeum&hl&&0.004& 7& 2&  7&  5&3&0&0.000&0.00& 0.000& 0.7&I (0/0/0.7, n = 2)\tabularnewline
Carex flacca&hl&&0.004&18& 5& 37& 32&3&0&0.000&0.00& 0.000& 0.7&I (0/0/0.7, n = 5)\tabularnewline
\midrule
Occuring only once& \multicolumn{14}{p{150mm}}{Alchemilla vulgaris s.str., Alopecurus pratensis, Astrantia major, Brachypodium pinnatum, Campanula trachelium, Cardaminopsis halleri, Carex montana, Carex muricata agg., Carex ornithopoda, Carex spicata, Carlina acaulis, Centaurea pseudophrygia, Cephalanthera longifolia, Chaerophyllum aromaticum, Clinopodium vulgare, Convolvulus arvensis, Dactylorhiza majalis, Dianthus carthusianorum, Filipendula vulgaris, Fraxinus excelsior, Galium aparine, Galium pumilum, Galium uliginosum, Geranium pyrenaicum, Hieracium lachenalii, Linum catharticum, Listera ovata, Lolium perenne, Medicago x varia, Myosotis scorpioides, Ornithogalum pyrenaicum ssp. sphaerocarpum, Phyteuma orbiculare, Plantago major, Polygala vulgaris, Potentilla reptans, Primula veris, Ranunculus nemorosus, Ranunculus polyanthemophyllus, Rumex obtusifolius, Salvia pratensis, Sedum sexangulare, Senecio germanicus, Silene nutans ssp. nutans, Symphytum officinale, Valeriana officinalis, Veronica serpyllifolia, Willemetia stipitata}\tabularnewline
\midrule
accuracy& \multicolumn{14}{p{150mm}}{.}\tabularnewline
altitude& \multicolumn{14}{p{150mm}}{368/682.5/\textbf{ 788.5 }/836/987}\tabularnewline
author& \multicolumn{14}{p{150mm}}{.}\tabularnewline
coverscale& \multicolumn{14}{p{150mm}}{.}\tabularnewline
e_coord& \multicolumn{14}{p{150mm}}{15.01316/15.345775/\textbf{ 15.379755 }/15.46579/15.61992}\tabularnewline
fels_antl& \multicolumn{14}{p{150mm}}{.}\tabularnewline
n_coord& \multicolumn{14}{p{150mm}}{47.76243/47.803985/\textbf{ 47.827845 }/47.920375/48.03626}\tabularnewline
nr_gl_ges& \multicolumn{14}{p{150mm}}{.}\tabularnewline
oevdat& \multicolumn{14}{p{150mm}}{.}\tabularnewline
releve_nr& \multicolumn{14}{p{150mm}}{14508/14535/\textbf{ 14558.5 }/14586/14655}\tabularnewline
schutt_ant& \multicolumn{14}{p{150mm}}{.}\tabularnewline
sp_count& \multicolumn{14}{p{150mm}}{.}\tabularnewline
surf_area& \multicolumn{14}{p{150mm}}{.}\tabularnewline
waypoint& \multicolumn{14}{p{150mm}}{526/527.5/\textbf{ 529 }/536/543}\tabularnewline
x_coord& \multicolumn{14}{p{150mm}}{15.013/15.3455/\textbf{ 15.38 }/15.4655/15.62}\tabularnewline
y_coord& \multicolumn{14}{p{150mm}}{47.762/47.804/\textbf{ 47.8275 }/47.9205/48.036}\tabularnewline
bezirk& \multicolumn{14}{p{150mm}}{Bruck an der Mur: 13; Lilienfeld: 7; Scheibbs: 5; St.Pölten-Land: 2; Sankt Pölten-Land: 1}\tabularnewline
date& \multicolumn{14}{p{150mm}}{2014-06-11: 6; 2014-06-18: 5; 2014-06-19: 4; 2014-06-10: 3; 2014-06-21: 3; 2014-06-05: 2; 2014-06-06: 2; 2014-06-09: 2; 2014-06-24: 1}\tabularnewline
diag_ms& \multicolumn{14}{p{150mm}}{Pastinaco-Arrhenatheretum alchemilletosum: 16; Filipendulo vulgaris-Arrhenatheretum: 6; Pastinaco-Arrhenatheretum brometosum erecti: 3; Ranunculo bulbosi-Arrhenatheretum: 2; Pastinaco-Arrhenatheretum typicum: 1}\tabularnewline
locality& \multicolumn{14}{p{150mm}}{Walstern: 9; Halltal: 4; Ötschergebiet: 3; Pielachtal: 3; Eisenwurzen: 2; Salzatal: 2; Taisental: 2; Traisental: 2; Türnitztal: 1}\tabularnewline
moss_ident& \multicolumn{14}{p{150mm}}{Y: 28}\tabularnewline
observer& \multicolumn{14}{p{150mm}}{Staudinger, Markus: 28}\tabularnewline
ordnung& \multicolumn{14}{p{150mm}}{Arrhenatheretalia: 28}\tabularnewline
orig_diag& \multicolumn{14}{p{150mm}}{Poo-Trisetetum: 14; Pastinaco-Arrhenatheretum: 3; Alchemillo-Arrhenatheretum: 2; Astrantio-Trisetetum: 2; Filipendulo-Arrhenatheretum: 2; Ranunculo bulbosi-Arrhenatheretum: 2; Anthoxantho-Agrostietum: 1; Arrhenatherion: 1; Festuco-Cynosuretum: 1}\tabularnewline
project& \multicolumn{14}{p{150mm}}{Bergmähwiesen NÖ: 15; Kartierung Halltall,Walstern: 13}\tabularnewline
quadrant& \multicolumn{14}{p{150mm}}{8158/3: 6; 8258/1: 6; 7959/4: 2; 8059/3: 2; 8157/2: 2; 8158/4: 2; 7958/4: 1; 8056/3: 1; 8057/1: 1; 8057/2: 1; 8057/4: 1; 8058/1: 1; 8058/3: 1; 8258/2: 1}\tabularnewline
region& \multicolumn{14}{p{150mm}}{NÖ Kalkvoralpen: 14; Steirische Kalk-Voralpen: 9; Steirische Kalkvoralpen: 4; NÖ Kalk-Voralpen: 1}\tabularnewline
remarks& \multicolumn{14}{p{150mm}}{am Gscheid SE Ulreichsberg, WP 370: 1; am Gscheid SE Ulreichsberg, WP 371: 1; am Ödboden im Högerwald, WP 410; ; ; ; : 1; am Ödboden im Högerwald, WP 411; Übergang zu Festuco-Cynosuretum, wohl nachbeweidet; ; ; ; : 1; bei Gösing an der Mariazeller Bahn, WP 397; : 1; beim Gh. Wastl am Wald S Puchenstuben, WP 395; 1 x gemäht im Jahr, seit 25 Jahren nicht mehr beweidet: 1; E Traisen, WP 418; ; ; ; ; : 1; E Traisen, WP 419; ; ; ; ; : 1; Kaltenmarkt N Pfaffenschlag, WP 381: 1; Lehengegend N Frankenfels, WP 377: 1; N Hubertussee, WP 490; ; ; ; ; ; : 1; oberes Halltal bei Terz, WP 543; ; : 1; oberes Halltal beim Lackenhof, WP 526; ; ; ; ; : 1; oberes Halltal beim Lackenhof, WP 529; sehr niederwüchsig; ; ; : 1; oberes Halltal, WP 517; ; ; ; ; ; : 1; Puchenstuben, WP 387: 1; Rechengraben, WP 479; sehr niederwüchsig, Übergang zu Festuco-Cynosuretum; ; ; ; ; ; ; : 1; Rechengraben, WP 484; niederwüchsig; ; ; ; ; ; : 1; Rotengraben NE Tradigist, WP 422; ; ; ; ; ; : 1; S des Hubertussee, WP 466; mit Wühlschäden; ; ; ; ; ; : 1; S Siebenbrunn, WP 413; etwas verbracht wirkend; ; ; ; : 1; Sattelgraben, WP 487; verbracht; ; ; ; ; ; : 1; Tal der Weißen Walster E des Hubertussee, WP 459; zum Begehungszeitpunkt keine Beweidung; ; ; ; ; : 1; Tal der Weißen Walster E des Hubertussee, WP 461; zum Begehungszeitpunkt keine Beweidung; ; ; ; ; : 1; Tal der Weißen Walster E des Hubertussee, WP 462; zum Begehungszeitpunkt keine Beweidung; ; ; ; ; : 1; Tal der Weißen Walster E des Hubertussee, WP 465; ; ; ; ; ; : 1; Taschlgrabenrotte, WP 409; homogene Fläche!!; ; ; : 1; W Höbarten am Schlagerboden E Scheibbs: 1}\tabularnewline
verband& \multicolumn{14}{p{150mm}}{Arrhenatherion: 28}\tabularnewline
\bottomrule
\end{longtable}

\newpage
%latex.default(as.matrix(tex.out[[i]]), file = file, append = TRUE,     caption = paste("Partion summary for cluster ", i, " consisting out of ",         table(partitioning(obj))[i], " plots.", ifelse(length(caption.text) >             0, paste(" ", caption.text, ".", sep = ""), ""),         sep = ""), rowname = NULL, booktabs = TRUE, longtable = TRUE,     lines.page = nrow(tex.out[[i]]), here = TRUE)%
\setlongtables\begin{longtable}{lllllllllllllll}\caption{Partion summary for cluster 2 consisting out of 131 plots.} \tabularnewline
\toprule
\multicolumn{1}{c}{taxon}&\multicolumn{1}{c}{layer}&\multicolumn{1}{c}{typical}&\multicolumn{1}{c}{stat}&\multicolumn{1}{c}{cons}&\multicolumn{1}{c}{cont}&\multicolumn{1}{c}{occu}&\multicolumn{1}{c}{out}&\multicolumn{1}{c}{spread}&\multicolumn{1}{c}{q0}&\multicolumn{1}{c}{q0.25}&\multicolumn{1}{c}{q0.5}&\multicolumn{1}{c}{q0.75}&\multicolumn{1}{c}{q1}&\multicolumn{1}{c}{summary}\tabularnewline
\midrule
\endfirsthead\caption[]{\em (continued)} \tabularnewline
\midrule
\multicolumn{1}{c}{taxon}&\multicolumn{1}{c}{layer}&\multicolumn{1}{c}{typical}&\multicolumn{1}{c}{stat}&\multicolumn{1}{c}{cons}&\multicolumn{1}{c}{cont}&\multicolumn{1}{c}{occu}&\multicolumn{1}{c}{out}&\multicolumn{1}{c}{spread}&\multicolumn{1}{c}{q0}&\multicolumn{1}{c}{q0.25}&\multicolumn{1}{c}{q0.5}&\multicolumn{1}{c}{q0.75}&\multicolumn{1}{c}{q1}&\multicolumn{1}{c}{summary}\tabularnewline
\midrule
\endhead
\midrule
\endfoot
\label{as.matrix}
Cerastium holosteoides&hl&&0.347&69& 90&119&29&3&0&0.0&0.3& 0.30& 0.7&IV (0/0.3/0.7, n = 90)\tabularnewline
Ranunculus acris&hl&&0.289&93&122&155&33&3&0&0.7&0.7& 4.00& 8.0&V (0/0.7/8, n = 122)\tabularnewline
Dactylis glomerata&hl&&0.238&92&121&158&37&3&0&0.7&4.0& 8.00&18.0&V (0/4/18, n = 121)\tabularnewline
Tragopogon orientalis&hl&&0.229&41& 54& 69&15&3&0&0.0&0.0& 0.30& 0.7&III (0/0/0.7, n = 54)\tabularnewline
Trifolium pratense&hl&&0.215&81&106&139&33&3&0&0.3&0.7& 3.00& 8.0&V (0/0.7/8, n = 106)\tabularnewline
Festuca nigrescens&hl&yes&0.207&73& 96&108&12&3&0&0.0&8.0&18.00&38.0&IV (0/8/38, n = 96)\tabularnewline
Dactylorhiza maculata s.lat. (inkl. fuchsii)&hl&yes&0.206&21& 27& 30& 3&2&0&0.0&0.0& 0.00& 0.3&II (0/0/0.3, n = 27)\tabularnewline
Lathyrus pratensis&hl&yes&0.196&43& 56& 66&10&3&0&0.0&0.0& 0.30& 0.7&III (0/0/0.7, n = 56)\tabularnewline
Campanula patula&hl&&0.190&38& 50& 61&11&3&0&0.0&0.0& 0.30& 0.7&II (0/0/0.7, n = 50)\tabularnewline
Stellaria graminea&hl&&0.186&37& 49& 58& 9&3&0&0.0&0.0& 0.30& 0.7&II (0/0/0.7, n = 49)\tabularnewline
Trisetum flavescens&hl&&0.161&79&104&142&38&3&0&0.7&4.0& 8.00&38.0&IV (0/4/38, n = 104)\tabularnewline
Rhinanthus alectorolophus&hl&&0.148&61& 80&101&21&3&0&0.0&0.3& 0.70& 8.0&IV (0/0.3/8, n = 80)\tabularnewline
Plantago lanceolata&hl&&0.145&66& 86&116&30&3&0&0.0&0.3& 0.70& 4.0&IV (0/0.3/4, n = 86)\tabularnewline
Alchemilla vulgaris agg.&hl&&0.130&76&100&124&24&3&0&0.3&0.7& 0.70& 8.0&IV (0/0.7/8, n = 100)\tabularnewline
Carex sylvatica&hl&&0.125&48& 63& 72& 9&3&0&0.0&0.0& 0.70& 4.0&III (0/0/4, n = 63)\tabularnewline
Trifolium repens&hl&&0.125&54& 71& 93&22&3&0&0.0&0.3& 0.70& 8.0&III (0/0.3/8, n = 71)\tabularnewline
Trollius europaeus&hl&yes&0.123&33& 43& 48& 5&3&0&0.0&0.0& 0.70& 8.0&II (0/0/8, n = 43)\tabularnewline
Agrostis capillaris&hl&yes&0.123&43& 56& 61& 5&3&0&0.0&0.0& 4.00&18.0&III (0/0/18, n = 56)\tabularnewline
Leontodon hispidus&hl&&0.123&59& 77& 96&19&3&0&0.0&0.7& 4.00&18.0&III (0/0.7/18, n = 77)\tabularnewline
Cirsium oleraceum&hl&&0.122&38& 50& 70&20&3&0&0.0&0.0& 0.70& 8.0&II (0/0/8, n = 50)\tabularnewline
Anthoxanthum odoratum&hl&yes&0.117&69& 91&106&15&3&0&0.0&0.7& 4.00&18.0&IV (0/0.7/18, n = 91)\tabularnewline
Carex pallescens&hl&yes&0.116&51& 67& 73& 6&3&0&0.0&0.3& 0.70& 4.0&III (0/0.3/4, n = 67)\tabularnewline
Primula elatior&hl&&0.115&11& 15& 18& 3&2&0&0.0&0.0& 0.00& 0.3&I (0/0/0.3, n = 15)\tabularnewline
Platanthera bifolia&hl&&0.115&11& 15& 17& 2&2&0&0.0&0.0& 0.00& 0.3&I (0/0/0.3, n = 15)\tabularnewline
Listera ovata&hl&&0.115&11& 15& 16& 1&2&0&0.0&0.0& 0.00& 0.3&I (0/0/0.3, n = 15)\tabularnewline
Phyteuma spicatum&hl&yes&0.115&18& 23& 23& 0&1&0&0.0&0.0& 0.00& 0.7&I (0/0/0.7, n = 23)\tabularnewline
Taraxacum officinale agg.&hl&&0.114&44& 57& 74&17&3&0&0.0&0.0& 0.70& 4.0&III (0/0/4, n = 57)\tabularnewline
Crepis biennis&hl&&0.112&45& 59& 80&21&3&0&0.0&0.0& 0.70& 8.0&III (0/0/8, n = 59)\tabularnewline
Festuca pratensis s.str.&hl&&0.104&65& 85&111&26&3&0&0.0&4.0& 6.00&38.0&IV (0/4/38, n = 85)\tabularnewline
Colchicum autumnale&hl&&0.100&56& 74& 98&24&3&0&0.0&0.3& 0.70& 8.0&III (0/0.3/8, n = 74)\tabularnewline
Myosotis nemorosa&hl&&0.099&19& 25& 26& 1&2&0&0.0&0.0& 0.00& 0.7&I (0/0/0.7, n = 25)\tabularnewline
Hypericum maculatum s.str.&hl&&0.097&62& 81& 97&16&3&0&0.0&0.3& 0.70&18.0&IV (0/0.3/18, n = 81)\tabularnewline
Lychnis flos-cuculi&hl&&0.093&21& 27& 31& 4&2&0&0.0&0.0& 0.00& 0.7&II (0/0/0.7, n = 27)\tabularnewline
Cynosurus cristatus&hl&&0.090&62& 81& 96&15&3&0&0.0&0.7& 4.00&38.0&IV (0/0.7/38, n = 81)\tabularnewline
Centaurea jacea&hl&&0.090&63& 82& 95&13&3&0&0.0&0.7& 1.85&18.0&IV (0/0.7/18, n = 82)\tabularnewline
Geranium sylvaticum&hl&&0.087&24& 31& 35& 4&2&0&0.0&0.0& 0.00& 8.0&II (0/0/8, n = 31)\tabularnewline
Heracleum sphondylium&hl&&0.086&32& 42& 54&12&3&0&0.0&0.0& 0.30& 4.0&II (0/0/4, n = 42)\tabularnewline
Pimpinella major&hl&&0.085&40& 53& 65&12&3&0&0.0&0.0& 0.70& 8.0&II (0/0/8, n = 53)\tabularnewline
Polygala vulgaris&hl&&0.084& 8& 11& 12& 1&2&0&0.0&0.0& 0.00& 0.3&I (0/0/0.3, n = 11)\tabularnewline
Poa trivialis&hl&&0.082&40& 52& 75&23&3&0&0.0&0.0& 3.00&18.0&II (0/0/18, n = 52)\tabularnewline
Prunella vulgaris&hl&&0.079&45& 59& 70&11&3&0&0.0&0.0& 0.30& 4.0&III (0/0/4, n = 59)\tabularnewline
Chaerophyllum hirsutum&hl&&0.078&43& 56& 64& 8&3&0&0.0&0.0& 0.70&18.0&III (0/0/18, n = 56)\tabularnewline
Cardaminopsis halleri&hl&&0.070&15& 20& 22& 2&3&0&0.0&0.0& 0.00& 0.7&I (0/0/0.7, n = 20)\tabularnewline
Lysimachia nemorum&hl&&0.065&12& 16& 16& 0&1&0&0.0&0.0& 0.00& 0.7&I (0/0/0.7, n = 16)\tabularnewline
Veronica chamaedrys&hl&&0.064&56& 74&102&28&3&0&0.0&0.3& 0.70& 4.0&III (0/0.3/4, n = 74)\tabularnewline
Ranunculus nemorosus&hl&&0.062&11& 15& 17& 2&3&0&0.0&0.0& 0.00& 0.7&I (0/0/0.7, n = 15)\tabularnewline
Galium pumilum&hl&&0.062&11& 15& 16& 1&2&0&0.0&0.0& 0.00& 0.7&I (0/0/0.7, n = 15)\tabularnewline
Astrantia major&hl&&0.062&27& 36& 37& 1&2&0&0.0&0.0& 0.30&18.0&II (0/0/18, n = 36)\tabularnewline
Achillea millefolium agg.&hl&&0.062&46& 60& 69& 9&3&0&0.0&0.0& 0.30& 3.0&III (0/0/3, n = 60)\tabularnewline
Vicia cracca&hl&&0.062&53& 69& 92&23&3&0&0.0&0.3& 0.30& 0.7&III (0/0.3/0.7, n = 69)\tabularnewline
Myosotis sylvatica&hl&&0.060&10& 13& 19& 6&3&0&0.0&0.0& 0.00& 0.7&I (0/0/0.7, n = 13)\tabularnewline
Thymus praecox&hl&&0.060&10& 13& 13& 0&1&0&0.0&0.0& 0.00& 0.7&I (0/0/0.7, n = 13)\tabularnewline
Briza media&hl&&0.060&44& 57& 68&11&2&0&0.0&0.0& 0.70&38.0&III (0/0/38, n = 57)\tabularnewline
Potentilla erecta&hl&&0.058&37& 49& 51& 2&2&0&0.0&0.0& 0.30& 3.0&II (0/0/3, n = 49)\tabularnewline
Lotus corniculatus&hl&&0.058&46& 60& 71&11&3&0&0.0&0.0& 0.30& 8.0&III (0/0/8, n = 60)\tabularnewline
Leucanthemum ircutianum&hl&&0.056&49& 64& 75&11&3&0&0.0&0.0& 0.30& 4.0&III (0/0/4, n = 64)\tabularnewline
Rumex acetosa&hl&&0.055&69& 91&115&24&3&0&0.0&0.3& 0.70& 8.0&IV (0/0.3/8, n = 91)\tabularnewline
Persicaria bistorta&hl&&0.054&23& 30& 36& 6&3&0&0.0&0.0& 0.00& 8.0&II (0/0/8, n = 30)\tabularnewline
Galium album s.str.&hl&&0.054&27& 35& 51&16&3&0&0.0&0.0& 0.30& 8.0&II (0/0/8, n = 35)\tabularnewline
Arrhenatherum elatius&hl&&0.052&21& 27& 43&16&3&0&0.0&0.0& 0.00&18.0&II (0/0/18, n = 27)\tabularnewline
Rhinanthus minor&hl&&0.052&27& 36& 45& 9&3&0&0.0&0.0& 0.30& 4.0&II (0/0/4, n = 36)\tabularnewline
Avenula pubescens&hl&&0.052&39& 51& 78&27&3&0&0.0&0.0& 3.00&18.0&II (0/0/18, n = 51)\tabularnewline
Carum carvi&hl&&0.051&33& 43& 58&15&3&0&0.0&0.0& 0.30& 8.0&II (0/0/8, n = 43)\tabularnewline
Aegopodium podagraria&hl&&0.047&37& 49& 66&17&3&0&0.0&0.0& 0.30& 4.0&II (0/0/4, n = 49)\tabularnewline
Veronica serpyllifolia&hl&&0.046& 5&  6&  8& 2&3&0&0.0&0.0& 0.00& 0.3&I (0/0/0.3, n = 6)\tabularnewline
Traunsteinera globosa&hl&&0.046& 5&  6&  6& 0&1&0&0.0&0.0& 0.00& 0.3&I (0/0/0.3, n = 6)\tabularnewline
Chaerophyllum aureum&hl&&0.046&33& 43& 55&12&3&0&0.0&0.0& 0.70&38.0&II (0/0/38, n = 43)\tabularnewline
Veratrum album&hl&&0.041&20& 26& 27& 1&2&0&0.0&0.0& 0.00& 8.0&I (0/0/8, n = 26)\tabularnewline
Linum catharticum&hl&&0.040& 8& 11& 12& 1&2&0&0.0&0.0& 0.00& 0.7&I (0/0/0.7, n = 11)\tabularnewline
Filipendula ulmaria&hl&&0.040&19& 25& 28& 3&2&0&0.0&0.0& 0.00& 4.0&I (0/0/4, n = 25)\tabularnewline
Tussilago farfara&hl&&0.039& 6&  8&  8& 0&1&0&0.0&0.0& 0.00& 0.7&I (0/0/0.7, n = 8)\tabularnewline
Plantago media&hl&&0.039& 6&  8&  8& 0&1&0&0.0&0.0& 0.00& 0.7&I (0/0/0.7, n = 8)\tabularnewline
Poa pratensis&hl&&0.039&15& 20& 38&18&3&0&0.0&0.0& 0.00& 8.0&I (0/0/8, n = 20)\tabularnewline
Dactylorhiza majalis&hl&&0.038& 4&  5&  6& 1&2&0&0.0&0.0& 0.00& 0.3&I (0/0/0.3, n = 5)\tabularnewline
Potentilla aurea&hl&&0.038& 6&  8&  8& 0&1&0&0.0&0.0& 0.00& 4.0&I (0/0/4, n = 8)\tabularnewline
Polygala amarella&hl&&0.038& 7&  9&  9& 0&1&0&0.0&0.0& 0.00& 0.7&I (0/0/0.7, n = 9)\tabularnewline
Cruciata laevipes&hl&&0.038&15& 20& 32&12&3&0&0.0&0.0& 0.00& 4.0&I (0/0/4, n = 20)\tabularnewline
Crepis paludosa&hl&&0.036&12& 16& 16& 0&1&0&0.0&0.0& 0.00& 4.0&I (0/0/4, n = 16)\tabularnewline
Willemetia stipitata&hl&&0.035&24& 31& 33& 2&3&0&0.0&0.0& 0.00& 8.0&II (0/0/8, n = 31)\tabularnewline
Lilium bulbiferum&hl&&0.031& 3&  4&  7& 3&3&0&0.0&0.0& 0.00& 0.3&I (0/0/0.3, n = 4)\tabularnewline
Plantago major&hl&&0.031& 3&  4&  5& 1&2&0&0.0&0.0& 0.00& 0.3&I (0/0/0.3, n = 4)\tabularnewline
Rumex obtusifolius&hl&&0.031& 9& 12& 17& 5&3&0&0.0&0.0& 0.00& 4.0&I (0/0/4, n = 12)\tabularnewline
Luzula multiflora s.str.&hl&&0.031&22& 29& 31& 2&2&0&0.0&0.0& 0.00& 4.0&II (0/0/4, n = 29)\tabularnewline
Phleum pratense&hl&&0.030&20& 26& 37&11&3&0&0.0&0.0& 0.00& 4.0&I (0/0/4, n = 26)\tabularnewline
Narcissus radiiflorus&hl&&0.029&15& 20& 23& 3&2&0&0.0&0.0& 0.00&18.0&I (0/0/18, n = 20)\tabularnewline
Bromus erectus&hl&&0.027& 4&  5&  9& 4&3&0&0.0&0.0& 0.00& 8.0&I (0/0/8, n = 5)\tabularnewline
Glechoma hederacea&hl&&0.027& 5&  7& 12& 5&3&0&0.0&0.0& 0.00& 0.7&I (0/0/0.7, n = 7)\tabularnewline
Trifolium medium&hl&&0.027&15& 19& 19& 0&1&0&0.0&0.0& 0.00& 3.0&I (0/0/3, n = 19)\tabularnewline
Vicia sepium&hl&&0.027&31& 41& 50& 9&3&0&0.0&0.0& 0.30& 0.7&II (0/0/0.7, n = 41)\tabularnewline
Symphytum officinale&hl&&0.026& 3&  4&  7& 3&3&0&0.0&0.0& 0.00& 0.7&I (0/0/0.7, n = 4)\tabularnewline
Valeriana dioica&hl&&0.025& 4&  5&  5& 0&1&0&0.0&0.0& 0.00& 0.7&I (0/0/0.7, n = 5)\tabularnewline
Carex flava&hl&&0.025& 4&  5&  5& 0&1&0&0.0&0.0& 0.00& 0.7&I (0/0/0.7, n = 5)\tabularnewline
Ranunculus repens&hl&&0.025&20& 26& 38&12&3&0&0.0&0.0& 0.00& 4.0&I (0/0/4, n = 26)\tabularnewline
Luzula luzuloides&hl&&0.024& 3&  4&  4& 0&1&0&0.0&0.0& 0.00& 4.0&I (0/0/4, n = 4)\tabularnewline
Silene dioica&hl&&0.024& 5&  6&  8& 2&2&0&0.0&0.0& 0.00& 0.7&I (0/0/0.7, n = 6)\tabularnewline
Rhinanthus glacialis&hl&&0.024&11& 14& 16& 2&2&0&0.0&0.0& 0.00& 4.0&I (0/0/4, n = 14)\tabularnewline
Alopecurus pratensis&hl&&0.024&14& 18& 24& 6&3&0&0.0&0.0& 0.00&18.0&I (0/0/18, n = 18)\tabularnewline
Galium uliginosum&hl&&0.023& 2&  3&  4& 1&2&0&0.0&0.0& 0.00& 0.3&I (0/0/0.3, n = 3)\tabularnewline
Carex leporina&hl&&0.023& 2&  3&  4& 1&2&0&0.0&0.0& 0.00& 0.7&I (0/0/0.7, n = 3)\tabularnewline
Anemone nemorosa&hl&&0.023& 2&  3&  3& 0&1&0&0.0&0.0& 0.00& 0.3&I (0/0/0.3, n = 3)\tabularnewline
Veronica officinalis&hl&&0.023& 2&  3&  3& 0&1&0&0.0&0.0& 0.00& 0.3&I (0/0/0.3, n = 3)\tabularnewline
Dentaria bulbifera&hl&&0.023& 2&  3&  3& 0&1&0&0.0&0.0& 0.00& 0.3&I (0/0/0.3, n = 3)\tabularnewline
Athyrium filix-femina&hl&&0.023& 2&  3&  3& 0&1&0&0.0&0.0& 0.00& 0.3&I (0/0/0.3, n = 3)\tabularnewline
Luzula pilosa&hl&&0.023& 2&  3&  3& 0&1&0&0.0&0.0& 0.00& 0.3&I (0/0/0.3, n = 3)\tabularnewline
Platanthera chlorantha&hl&&0.023& 2&  3&  3& 0&1&0&0.0&0.0& 0.00& 0.3&I (0/0/0.3, n = 3)\tabularnewline
Euphrasia rostkoviana&hl&&0.023& 2&  3&  3& 0&1&0&0.0&0.0& 0.00& 0.3&I (0/0/0.3, n = 3)\tabularnewline
Veronica arvensis&hl&&0.023& 5&  7& 16& 9&3&0&0.0&0.0& 0.00& 0.3&I (0/0/0.3, n = 7)\tabularnewline
Campanula scheuchzeri&hl&&0.023& 5&  7& 11& 4&2&0&0.0&0.0& 0.00& 0.3&I (0/0/0.3, n = 7)\tabularnewline
Carex montana&hl&&0.023& 6&  8&  9& 1&2&0&0.0&0.0& 0.00& 8.0&I (0/0/8, n = 8)\tabularnewline
Carex flacca&hl&&0.023&24& 31& 37& 6&3&0&0.0&0.0& 0.00&18.0&II (0/0/18, n = 31)\tabularnewline
Potentilla anserina&hl&&0.022& 3&  4&  4& 0&1&0&0.0&0.0& 0.00& 0.7&I (0/0/0.7, n = 4)\tabularnewline
Cirsium erisithales&hl&&0.022& 5&  7&  7& 0&1&0&0.0&0.0& 0.00& 4.0&I (0/0/4, n = 7)\tabularnewline
Centaurea scabiosa&hl&&0.022& 7&  9& 14& 5&2&0&0.0&0.0& 0.00& 8.0&I (0/0/8, n = 9)\tabularnewline
Holcus lanatus&hl&&0.021& 4&  5& 10& 5&3&0&0.0&0.0& 0.00&18.0&I (0/0/18, n = 5)\tabularnewline
Stachys alpina&hl&&0.021& 4&  5&  8& 3&3&0&0.0&0.0& 0.00& 0.7&I (0/0/0.7, n = 5)\tabularnewline
Carlina acaulis&hl&&0.021& 8& 11& 12& 1&2&0&0.0&0.0& 0.00& 3.0&I (0/0/3, n = 11)\tabularnewline
Leucanthemum vulgare&hl&&0.021&13& 17& 19& 2&2&0&0.0&0.0& 0.00& 4.0&I (0/0/4, n = 17)\tabularnewline
Ajuga reptans&hl&&0.021&16& 21& 26& 5&2&0&0.0&0.0& 0.00& 0.7&I (0/0/0.7, n = 21)\tabularnewline
Myosotis arvensis&hl&&0.020& 5&  6& 11& 5&3&0&0.0&0.0& 0.00& 0.3&I (0/0/0.3, n = 6)\tabularnewline
Petasites hybridus&hl&&0.020& 8& 10& 12& 2&2&0&0.0&0.0& 0.00&18.0&I (0/0/18, n = 10)\tabularnewline
Cirsium rivulare&hl&&0.019& 4&  5&  5& 0&1&0&0.0&0.0& 0.00& 4.0&I (0/0/4, n = 5)\tabularnewline
Bellis perennis&hl&&0.019&12& 16& 24& 8&3&0&0.0&0.0& 0.00& 3.0&I (0/0/3, n = 16)\tabularnewline
Alchemilla vulgaris s.str.&hl&&0.017& 2&  3&  5& 2&3&0&0.0&0.0& 0.00& 4.0&I (0/0/4, n = 3)\tabularnewline
Gymnadenia conopsea&hl&&0.017& 3&  4&  7& 3&2&0&0.0&0.0& 0.00& 0.7&I (0/0/0.7, n = 4)\tabularnewline
Crepis mollis&hl&&0.017& 3&  4&  6& 2&2&0&0.0&0.0& 0.00& 0.7&I (0/0/0.7, n = 4)\tabularnewline
Polygonatum verticillatum&hl&&0.017& 3&  4&  4& 0&1&0&0.0&0.0& 0.00& 0.7&I (0/0/0.7, n = 4)\tabularnewline
Acer pseudoplatanus&hl&&0.017& 3&  4&  4& 0&1&0&0.0&0.0& 0.00& 0.7&I (0/0/0.7, n = 4)\tabularnewline
Hieracium lachenalii&hl&&0.015& 2&  2&  3& 1&2&0&0.0&0.0& 0.00& 0.3&I (0/0/0.3, n = 2)\tabularnewline
Asarum europaeum&hl&&0.015& 2&  2&  2& 0&1&0&0.0&0.0& 0.00& 0.3&I (0/0/0.3, n = 2)\tabularnewline
Molinia caerulea agg.&hl&&0.015& 2&  2&  2& 0&1&0&0.0&0.0& 0.00& 0.7&I (0/0/0.7, n = 2)\tabularnewline
Ranunculus auricomus s.lat.&hl&&0.015& 2&  2&  2& 0&1&0&0.0&0.0& 0.00& 0.3&I (0/0/0.3, n = 2)\tabularnewline
Scirpus sylvaticus&hl&&0.015& 2&  2&  2& 0&1&0&0.0&0.0& 0.00& 0.7&I (0/0/0.7, n = 2)\tabularnewline
Viola hirta&hl&&0.015& 2&  2&  2& 0&1&0&0.0&0.0& 0.00& 0.3&I (0/0/0.3, n = 2)\tabularnewline
Helleborus niger&hl&&0.015& 2&  2&  2& 0&1&0&0.0&0.0& 0.00& 0.3&I (0/0/0.3, n = 2)\tabularnewline
Allium carinatum&hl&&0.015& 2&  2&  2& 0&1&0&0.0&0.0& 0.00& 0.3&I (0/0/0.3, n = 2)\tabularnewline
Sesleria albicans&hl&&0.015& 2&  2&  2& 0&1&0&0.0&0.0& 0.00& 0.7&I (0/0/0.7, n = 2)\tabularnewline
Trifolium montanum&hl&&0.015& 2&  2&  2& 0&1&0&0.0&0.0& 0.00& 0.3&I (0/0/0.3, n = 2)\tabularnewline
Silene vulgaris&hl&&0.015& 2&  2&  2& 0&1&0&0.0&0.0& 0.00& 0.3&I (0/0/0.3, n = 2)\tabularnewline
Carduus personata&hl&&0.015& 2&  2&  2& 0&1&0&0.0&0.0& 0.00& 0.3&I (0/0/0.3, n = 2)\tabularnewline
Knautia arvensis&hl&&0.015&11& 15& 23& 8&2&0&0.0&0.0& 0.00& 8.0&I (0/0/8, n = 15)\tabularnewline
Carex panicea&hl&&0.015&14& 18& 18& 0&1&0&0.0&0.0& 0.00& 8.0&I (0/0/8, n = 18)\tabularnewline
Ranunculus bulbosus&hl&&0.014& 2&  3&  8& 5&3&0&0.0&0.0& 0.00& 0.7&I (0/0/0.7, n = 3)\tabularnewline
Phyteuma orbiculare&hl&&0.014& 2&  3&  4& 1&2&0&0.0&0.0& 0.00& 0.7&I (0/0/0.7, n = 3)\tabularnewline
Sanguisorba minor&hl&&0.014& 2&  3&  3& 0&1&0&0.0&0.0& 0.00& 0.7&I (0/0/0.7, n = 3)\tabularnewline
Gentiana asclepiadea&hl&&0.014& 2&  3&  3& 0&1&0&0.0&0.0& 0.00& 0.7&I (0/0/0.7, n = 3)\tabularnewline
Trifolium dubium&hl&&0.014& 2&  3&  3& 0&1&0&0.0&0.0& 0.00& 0.7&I (0/0/0.7, n = 3)\tabularnewline
Angelica sylvestris&hl&&0.014& 2&  3&  3& 0&1&0&0.0&0.0& 0.00& 0.7&I (0/0/0.7, n = 3)\tabularnewline
Carex pilulifera&hl&&0.014& 2&  3&  3& 0&1&0&0.0&0.0& 0.00& 4.0&I (0/0/4, n = 3)\tabularnewline
Nardus stricta&hl&&0.014& 5&  6&  6& 0&1&0&0.0&0.0& 0.00&18.0&I (0/0/18, n = 6)\tabularnewline
Cirsium arvense&hl&&0.013& 5&  7& 12& 5&3&0&0.0&0.0& 0.00& 4.0&I (0/0/4, n = 7)\tabularnewline
Senecio subalpinus&hl&&0.013& 6&  8& 10& 2&2&0&0.0&0.0& 0.00& 8.0&I (0/0/8, n = 8)\tabularnewline
Geranium phaeum&hl&&0.012& 2&  3&  7& 4&3&0&0.0&0.0& 0.00& 8.0&I (0/0/8, n = 3)\tabularnewline
Knautia maxima&hl&&0.012& 3&  4&  4& 0&1&0&0.0&0.0& 0.00& 4.0&I (0/0/4, n = 4)\tabularnewline
Salvia verticillata&hl&&0.012& 5&  6& 10& 4&2&0&0.0&0.0& 0.00& 4.0&I (0/0/4, n = 6)\tabularnewline
Medicago lupulina&hl&&0.012&11& 15& 25&10&3&0&0.0&0.0& 0.00& 0.7&I (0/0/0.7, n = 15)\tabularnewline
Centaurea pseudophrygia&hl&&0.011& 2&  2&  3& 1&2&0&0.0&0.0& 0.00& 0.7&I (0/0/0.7, n = 2)\tabularnewline
Myosotis scorpioides&hl&&0.011& 2&  2&  3& 1&2&0&0.0&0.0& 0.00& 0.7&I (0/0/0.7, n = 2)\tabularnewline
Geum urbanum&hl&&0.011& 2&  2&  3& 1&2&0&0.0&0.0& 0.00& 0.7&I (0/0/0.7, n = 2)\tabularnewline
Leontodon autumnalis&hl&&0.011& 2&  2&  3& 1&2&0&0.0&0.0& 0.00& 0.7&I (0/0/0.7, n = 2)\tabularnewline
Alchemilla monticola&hl&&0.011& 2&  2&  2& 0&1&0&0.0&0.0& 0.00& 0.7&I (0/0/0.7, n = 2)\tabularnewline
Sanguisorba officinalis&hl&&0.011& 2&  2&  2& 0&1&0&0.0&0.0& 0.00& 0.7&I (0/0/0.7, n = 2)\tabularnewline
Fragaria vesca&hl&&0.011& 2&  2&  2& 0&1&0&0.0&0.0& 0.00& 0.7&I (0/0/0.7, n = 2)\tabularnewline
Hieracium pilosella&hl&&0.011& 2&  2&  2& 0&1&0&0.0&0.0& 0.00& 0.7&I (0/0/0.7, n = 2)\tabularnewline
Stellaria nemorum s.str.&hl&&0.011& 2&  2&  2& 0&1&0&0.0&0.0& 0.00& 0.7&I (0/0/0.7, n = 2)\tabularnewline
Laserpitium latifolium&hl&&0.011& 2&  2&  2& 0&1&0&0.0&0.0& 0.00& 0.7&I (0/0/0.7, n = 2)\tabularnewline
Hieracium lactucella&hl&&0.011& 2&  2&  2& 0&1&0&0.0&0.0& 0.00& 0.7&I (0/0/0.7, n = 2)\tabularnewline
Senecio ovatus&hl&&0.011& 2&  2&  2& 0&1&0&0.0&0.0& 0.00& 0.7&I (0/0/0.7, n = 2)\tabularnewline
Galeopsis tetrahit&hl&&0.011& 2&  2&  2& 0&1&0&0.0&0.0& 0.00& 0.7&I (0/0/0.7, n = 2)\tabularnewline
Cirsium palustre&hl&&0.011& 2&  2&  2& 0&1&0&0.0&0.0& 0.00& 0.7&I (0/0/0.7, n = 2)\tabularnewline
Deschampsia cespitosa&hl&&0.011& 7&  9& 11& 2&2&0&0.0&0.0& 0.00&18.0&I (0/0/18, n = 9)\tabularnewline
Equisetum arvense&hl&&0.011&10& 13& 15& 2&2&0&0.0&0.0& 0.00& 0.7&I (0/0/0.7, n = 13)\tabularnewline
Aconitum napellus s.str.&hl&&0.010& 4&  5&  5& 0&1&0&0.0&0.0& 0.00& 4.0&I (0/0/4, n = 5)\tabularnewline
Geum rivale&hl&&0.010& 8& 10& 13& 3&2&0&0.0&0.0& 0.00& 0.7&I (0/0/0.7, n = 10)\tabularnewline
Brachypodium pinnatum&hl&&0.009& 2&  3&  5& 2&3&0&0.0&0.0& 0.00& 8.0&I (0/0/8, n = 3)\tabularnewline
Equisetum palustre&hl&&0.009& 2&  3&  3& 0&1&0&0.0&0.0& 0.00& 4.0&I (0/0/4, n = 3)\tabularnewline
Hypochaeris radicata&hl&&0.009& 2&  3&  3& 0&1&0&0.0&0.0& 0.00& 3.0&I (0/0/3, n = 3)\tabularnewline
Aconitum napellus agg.&hl&&0.009& 2&  2&  2& 0&1&0&0.0&0.0& 0.00& 3.0&I (0/0/3, n = 2)\tabularnewline
Clinopodium vulgare&hl&&0.009& 5&  6&  8& 2&3&0&0.0&0.0& 0.00& 8.0&I (0/0/8, n = 6)\tabularnewline
Buphthalmum salicifolium&hl&&0.007& 5&  7&  9& 2&2&0&0.0&0.0& 0.00& 0.7&I (0/0/0.7, n = 7)\tabularnewline
Carex hirta&hl&&0.006& 4&  5& 10& 5&2&0&0.0&0.0& 0.00& 0.7&I (0/0/0.7, n = 5)\tabularnewline
Elymus repens&hl&&0.004& 3&  4&  6& 2&2&0&0.0&0.0& 0.00& 4.0&I (0/0/4, n = 4)\tabularnewline
Arabis hirsuta&hl&&0.002& 2&  3&  9& 6&2&0&0.0&0.0& 0.00& 0.3&I (0/0/0.3, n = 3)\tabularnewline
Origanum vulgare&hl&&0.002& 2&  3&  5& 2&2&0&0.0&0.0& 0.00& 0.7&I (0/0/0.7, n = 3)\tabularnewline
\midrule
Occuring only once& \multicolumn{14}{p{150mm}}{Anthyllis vulneraria, Arenaria serpyllifolia s.str., Betonica officinalis, Brachypodium rupestre, Bromus hordeaceus, Calluna vulgaris, Caltha palustris, Campanula trachelium, Cardamine hirsuta, Carduus acanthoides, Carduus defloratus s.lat., Carex caryophyllea, Carex echinata, Carex flacca, Carex paniculata, Carex pulicaris, Carex vulpina, Crepis aurea, Crepis praemorsa, Cruciata glabra, Daucus carota, Deschampsia cespitosa ssp. gaudinii, Dryopteris filix-mas, Epilobium alpestre, Epilobium montanum, Equisetum sylvaticum, Euphorbia cyparissias, Galium album s.lat., Galium anisophyllon, Galium austriacum, Gentiana verna, Geranium columbinum, Geranium pyrenaicum, Heracleum austriacum, Hieracium murorum, Hieracium piloselloides, Hypericum montanum, Hypericum perforatum, Juncus articulatus, Juncus filiformis, Linaria vulgaris, Lolium multiflorum, Luzula campestris, Lysimachia vulgaris, Melampyrum pratense, Melampyrum sylvaticum, Mentha longifolia, Mercurialis perennis, Orchis mascula, Paris quadrifolia, Picea abies, Picris hieracioides, Pimpinella saxifraga s.str., Poa alpina, Poa annua, Poa supina, Polygonatum multiflorum, Potentilla recta, Potentilla reptans, Primula veris, Ranunculus aconitifolius, Rosa pendulina, Salvia glutinosa, Scabiosa lucida, Scorzonera humilis, Scutellaria galericulata, Thymus pulegioides, Urtica dioica, Vaccinium myrtillus, Valeriana officinalis}\tabularnewline
\midrule
accuracy& \multicolumn{14}{p{150mm}}{.}\tabularnewline
altitude& \multicolumn{14}{p{150mm}}{567/795.5/\textbf{ 848 }/911.5/1030}\tabularnewline
author& \multicolumn{14}{p{150mm}}{.}\tabularnewline
coverscale& \multicolumn{14}{p{150mm}}{.}\tabularnewline
e_coord& \multicolumn{14}{p{150mm}}{15.02628/15.348935/\textbf{ 15.38462 }/15.41786/15.47318}\tabularnewline
fels_antl& \multicolumn{14}{p{150mm}}{.}\tabularnewline
n_coord& \multicolumn{14}{p{150mm}}{47.74587/47.762195/\textbf{ 47.76898 }/47.82408/47.99713}\tabularnewline
nr_gl_ges& \multicolumn{14}{p{150mm}}{.}\tabularnewline
oevdat& \multicolumn{14}{p{150mm}}{.}\tabularnewline
releve_nr& \multicolumn{14}{p{150mm}}{14494/14545.5/\textbf{ 14644 }/14685.5/14721}\tabularnewline
schutt_ant& \multicolumn{14}{p{150mm}}{.}\tabularnewline
sp_count& \multicolumn{14}{p{150mm}}{.}\tabularnewline
surf_area& \multicolumn{14}{p{150mm}}{.}\tabularnewline
waypoint& \multicolumn{14}{p{150mm}}{520/547/\textbf{ 570.5 }/591/612}\tabularnewline
x_coord& \multicolumn{14}{p{150mm}}{15.026/15.349/\textbf{ 15.385 }/15.418/15.473}\tabularnewline
y_coord& \multicolumn{14}{p{150mm}}{47.746/47.762/\textbf{ 47.769 }/47.824/47.997}\tabularnewline
bezirk& \multicolumn{14}{p{150mm}}{Bruck an der Mur: 96; Scheibbs: 22; Lilienfeld: 9; Sankt Pölten-Land: 2; St.Pölten-Land: 2}\tabularnewline
date& \multicolumn{14}{p{150mm}}{2014-06-25: 19; 2014-06-24: 17; 2014-06-27: 15; 2014-06-09: 12; 2014-06-18: 11; 2014-06-26: 11; 2014-06-10: 10; 2014-06-21: 10; 2014-06-19: 8; 2014-06-05: 7; 2014-06-28: 5; 2014-06-06: 4; 2014-06-04: 2}\tabularnewline
diag_ms& \multicolumn{14}{p{150mm}}{Lolio-Cynosuretum (incl. Festuco-Cynosuretum): 34; 'Avenulo-Festucetum chaerophylletosum aurei': 32; 'Astrantio-Caricetum montanae': 18; Crepido-Festucetum (incl. Crepido-Cynosuretum): 16; 'Avenulo-Fetucetum brizetosum mediae': 13; Lychnido-Festucetum rubrae: 9; 'Avenulo-Festucetum aegopodietosum podagrariae': 8; Anthoxantho-Agrostietum': 1}\tabularnewline
locality& \multicolumn{14}{p{150mm}}{Halltal: 77; Walstern: 19; Ötschergebiet: 13; Eisenwurzen: 9; Salzatal: 6; Karnerbachtal: 3; Pielachtal: 3; Ötschergegend: 1}\tabularnewline
moss_ident& \multicolumn{14}{p{150mm}}{Y: 131}\tabularnewline
observer& \multicolumn{14}{p{150mm}}{Staudinger, Markus: 131}\tabularnewline
ordnung& \multicolumn{14}{p{150mm}}{Arrhenatheretalia: 131}\tabularnewline
orig_diag& \multicolumn{14}{p{150mm}}{Poo-Trisetetum: 49; Festuco-Cynosuretum: 41; Astrantio-Trisetetum: 16; Campanulo-Agrostietum: 8; Bromion: 5; Anthoxantho-Agrostietum: 3; Astrantio-Festucetum: 3; Alchemillo-Arrhenatheretum: 2; Arrhenatherion: 1; Astrantio-Brometum: 1; Nardion: 1; Petasition officinalis: 1}\tabularnewline
project& \multicolumn{14}{p{150mm}}{Kartierung Halltall,Walstern: 96; Bergmähwiesen NÖ: 35}\tabularnewline
quadrant& \multicolumn{14}{p{150mm}}{8258/1: 56; 8258/2: 25; 8158/3: 12; 8158/4: 9; 8057/4: 7; 8157/2: 7; 8057/1: 5; 8056/3: 3; 8258/3: 3; 8057/2: 2; 8058/1: 2}\tabularnewline
region& \multicolumn{14}{p{150mm}}{Steirische Kalkvoralpen: 77; NÖ Kalkvoralpen: 35; Steirische Kalk-Voralpen: 19}\tabularnewline
remarks& \multicolumn{14}{p{150mm}}{am Bölzenberg N Pfaffenschlag, WP 382: 1; am Gscheid SE Ulreichsberg, WP 360: 1; am Michelbühel NE Ulreichsberg, WP 352: 1; beim Gh. Wastl am Wald S Puchenstuben, WP 393; 1 x gemäht im Jahr, seit 25 Jahren nicht mehr beweidet: 1; beim Gh. Wastl am Wald S Puchenstuben, WP 394; 1 x gemäht im Jahr, seit 25 Jahren nicht mehr beweidet: 1; beim Schüsseleck E Gösing an der Mariazeller Bahn, WP 399; sehr niederwüchisge Sommerweide; : 1; beim Schüsseleck E Gösing an der Mariazeller Bahn, WP 402; ; : 1; beim Schüsseleck E Gösing an der Mariazeller Bahn, WP 403; ; : 1; Bölzenberg N Pfaffenschlag, WP 379: 1; Brandeben N Puchenstuben, WP 388: 1; Brandeben N Puchenstuben, WP 389: 1; Brandeben N Puchenstuben, WP 390: 1; Brandeben N Puchenstuben, WP 391: 1; Brandeben N Puchenstuben, WP 392: 1; ehemalige Weide am Gscheid SE Ulreichsberg, WP 366: 1; Feichtenbachgraben am Gscheid SE Ulreichsberg, WP 373: 1; Hochalm N Pfaffenschlag, WP 380: 1; Hochstadelberg E Gösing an der Mariazeller Bahn, WP 404; verbracht; ; : 1; Hollenstein am Schlagerboden E Scheibbs; : 1; Hollenstein am Schlagerboden E Scheibbs; gemäht: 1; Hollenstein E St. Anton an der Jeßnitz, WP 384: 1; Lehengegend N Frankenfels, WP 378: 1; N des Hubertussee, WP 468; eher fett und artenarm; ; ; ; ; ; ; : 1; N des Hubertussee, WP 469; ; ; ; ; ; ; : 1; N des Hubertussee, WP 471; ; ; ; ; ; ; : 1; N des Hubertussee, WP 472; ; ; ; ; ; ; : 1; N Hubertussee, WP 492; ; ; ; ; ; : 1; Nattersbachtal SE Puchenstuben, WP 405; ; ; : 1; NE Ulreichsberg, WP 346: 1; NE Ulreichsberg, WP 350: 1; oberes Halltal bei Terz, WP 539; ; : 1; oberes Halltal bei Terz, WP 541; ; : 1; oberes Halltal bei Terz, WP 542; ; : 1; oberes Halltal bei Terz, WP 545; ; : 1; oberes Halltal bei Terz, WP 546; ; : 1; oberes Halltal bei Terz, WP 547; nicht beweidet; : 1; oberes Halltal bei Terz, WP 548; Brache mit aufkommenden Fichten; : 1; oberes Halltal bei Terz, WP 549; etwas verbracht wirkend; : 1; oberes Halltal bei Terz, WP 550; Fettwiese; : 1; oberes Halltal beim Frühwirt, WP 536; sehr niederwüchsig; ; : 1; oberes Halltal beim Frühwirt, WP 537; ; : 1; oberes Halltal beim Lackenhof, WP 528; sehr fette Wiese; ; ; : 1; oberes Halltal beim Lackenhof, WP 530; ; ; : 1; oberes Halltal beim Lackenhof, WP 531; beweidet; ; : 1; oberes Halltal beim Lackenhof, WP 532; sehr extensiv beweidet; ; ; : 1; oberes Halltal, beim Greierhof, WP 582 Fettwiese: 1; oberes Halltal, beim Hönbichler, WP 569: 1; oberes Halltal, Böschung beim Hönbichler, WP 580: 1; oberes Halltal, Höllgraben, WP 533; Talbodenfettwiese; ; ; : 1; oberes Halltal, Höllgraben, WP 563; Fettwiese; : 1; oberes Halltal, Höllgraben, WP 564; niederwüchsig; : 1; oberes Halltal, Höllgraben, WP 565; : 1; oberes Halltal, Höllgraben, WP 566; : 1; oberes Halltal, Höllgraben, WP 567; steile Böschung, im Herbst nachbeweidet: 1; oberes Halltal, Kuhgraben, WP 554; ; : 1; oberes Halltal, Lackengraben, WP 579 Weide: 1; oberes Halltal, WP 519; typische Goldhaferfettwiese der Hänge; ; ; ; ; ; : 1; oberes Halltal, WP 520; typische Goldhaferfettwiese der Hänge, nachbeweidet; ; ; ; ; ; : 1; oberes Halltal, WP 521; ; ; ; ; ; : 1; oberes Halltal, WP 522; ; ; ; ; ; : 1; oberes Halltal, WP 523; ; ; ; ; ; : 1; oberes Halltal, WP 525; ; ; ; ; : 1; oberes Halltal, WP 535; ; ; : 1; oberes Halltal, WP 553; etwas verbracht wirkend; : 1; oberes Halltal, WP 556; sehr hochwüchsig; : 1; oberes Halltal, WP 557; sehr hochwüchsige Fettwiese; : 1; oberes Halltal, WP 558; Weide; : 1; oberes Halltal, WP 559; dichte und hohe Grasschicht; : 1; oberes Halltal, WP 560; Fettwiese; : 1; oberes Halltal, WP 561; nicht beweidet zur Aufnahme; : 1; oberes Halltal, WP 562; nicht beweidet zur Aufnahme; : 1; oberes Halltal, WP 572 Brache: 1; oberes Halltal, WP 573: 1; oberes Halltal, WP 574 Brache, mit Weidegangeln: 1; oberes Halltal, WP 575 derzeit nicht beweidet: 1; oberes Halltal, WP 576 schattige Waldwiese: 1; oberes Halltal, WP 577 ehemalige Weide, seit etwa 15 Jahren nicht mehr genutzt mit etwa 3m hohen Fichten: 1; oberes Halltal, WP 577 niederwüchsig: 1; oberes Halltal, WP 581: 1; oberes Halltal, WP 602: 1; oberes Halltal, WP 603: 1; oberes Halltal, WP 604 sehr magere niederwüchsige Wiese, nicht beweidet: 1; oberes Halltal, WP 607 Weide: 1; oberes Halltal, WP 609 Weide: 1; Puchenstuben, WP 386: 1; Rechengraben, WP 476; sehr dicht stehende Fettwiese; ; ; ; ; ; ; : 1; Rechengraben, WP 478; verbracht; ; ; ; ; ; ; : 1; Rechengraben, WP 480; sehr niederwüchsig; ; ; ; ; ; : 1; Rechengraben, WP 481; verbracht; ; ; ; ; ; : 1; Rechengraben, WP 482; verbracht; ; ; ; ; ; : 1; Rechengraben, WP 483; niederwüchsig; ; ; ; ; ; : 1; Rechengraben, WP 485; eher Saumgesellschaft; ; ; ; ; ; : 1; S des Hubertussee, WP 467; ; ; ; ; ; ; : 1; S Michelbühel NE Ulreichsberg, WP 355: 1; SE Fadental im Otterbachtal, WP 451; zum Begehungszeitpunkt keine Beweidung; ; ; ; ; : 1; Tal der Weißen Walster E des Hubertussee, WP 452; zum Begehungszeitpunkt keine Beweidung; ; ; ; ; : 1; Tal der Weißen Walster E des Hubertussee, WP 453; zum Begehungszeitpunkt keine Beweidung; ; ; ; ; : 1; Tal der Weißen Walster E des Hubertussee, WP 455; zum Begehungszeitpunkt keine Beweidung; ; ; ; ; : 1; Tal der Weißen Walster E des Hubertussee, WP 458; zum Begehungszeitpunkt keine Beweidung; ; ; ; ; : 1; Tal der Weißen Walster E des Hubertussee, WP 460; zum Begehungszeitpunkt keine Beweidung; ; ; ; ; : 1; Taschlgrabenrotte, WP 408; homogene Fläche, nicht verbracht!!; ; ; : 1; unteres Halltal beim Thalerhof, WP 514; Talbodenfettwiese; ; ; ; ; : 1; unteres Halltal beim Thalerhof, WP 515; Talbodenfettwiese; ; ; ; ; : 1; unteres Halltal, Gracheralm, WP 611 derzeit keine Beweidung: 1; unteres Halltal, Gracheralm, WP 612: 1; unteres Halltal, Greiergraben, WP 583 etwas verbracht wirkend: 1; unteres Halltal, Mooshuben, WP 590 Fettwiese: 1; unteres Halltal, Mooshuben, WP 591 Weide: 1; unteres Halltal, Mooshuben, WP 592 Fettwiese: 1; unteres Halltal, Mooshuben, WP 593: 1; unteres Halltal, Mooshuben, WP 594: 1; unteres Halltal, Mooshuben, WP 595: 1; unteres Halltal, Mooshuben, WP 596 feuchte Fettweide: 1; unteres Halltal, Mooshuben, WP 597 Fettwiese: 1; unteres Halltal, Mooshuben, WP 598: 1; unteres Halltal, Mooshuben, WP 599 Weide: 1; unteres Halltal, Mooshuben, WP 600 Brache mit aufkommenden Fichten: 1; unteres Halltal, Mooshuben, WP 601: 1; unteres Halltal, WP 584: 1; unteres Halltal, WP 585: 1; unteres Halltal, WP 586: 1; unteres Halltal, WP 588: 1; unteres Halltal, WP 589: 1; unteres Halltal, WP 606 verbracht: 1; verbrachte Goldhaferwiese am Gscheid SE Ulreichsberg, WP 368: 1; verbrachte Weide am Gscheid SE Ulreichsberg, WP 362: 1; W Höbarten am Schlagerboden E Scheibbs: 1; W Höbarten am Schlagerboden E Scheibbs; Fläche beweidet: 1; W Höbarten am Schlagerboden E Scheibbs; Fläche gemäht, nicht beweidet: 1; Waizgraben, Falkensteinrotte, WP 407; ; ; : 1; Zw. Schüsseleck und Hochstadelberg E Gösing an der Mariazeller Bahn, WP 398; schattige Waldwiese; : 1}\tabularnewline
verband& \multicolumn{14}{p{150mm}}{Cynosurion: 131}\tabularnewline
\bottomrule
\end{longtable}

\newpage
%latex.default(as.matrix(tex.out[[i]]), file = file, append = TRUE,     caption = paste("Partion summary for cluster ", i, " consisting out of ",         table(partitioning(obj))[i], " plots.", ifelse(length(caption.text) >             0, paste(" ", caption.text, ".", sep = ""), ""),         sep = ""), rowname = NULL, booktabs = TRUE, longtable = TRUE,     lines.page = nrow(tex.out[[i]]), here = TRUE)%
\setlongtables\begin{longtable}{lllllllllllllll}\caption{Partion summary for cluster 3 consisting out of 11 plots.} \tabularnewline
\toprule
\multicolumn{1}{c}{taxon}&\multicolumn{1}{c}{layer}&\multicolumn{1}{c}{typical}&\multicolumn{1}{c}{stat}&\multicolumn{1}{c}{cons}&\multicolumn{1}{c}{cont}&\multicolumn{1}{c}{occu}&\multicolumn{1}{c}{out}&\multicolumn{1}{c}{spread}&\multicolumn{1}{c}{q0}&\multicolumn{1}{c}{q0.25}&\multicolumn{1}{c}{q0.5}&\multicolumn{1}{c}{q0.75}&\multicolumn{1}{c}{q1}&\multicolumn{1}{c}{summary}\tabularnewline
\midrule
\endfirsthead\caption[]{\em (continued)} \tabularnewline
\midrule
\multicolumn{1}{c}{taxon}&\multicolumn{1}{c}{layer}&\multicolumn{1}{c}{typical}&\multicolumn{1}{c}{stat}&\multicolumn{1}{c}{cons}&\multicolumn{1}{c}{cont}&\multicolumn{1}{c}{occu}&\multicolumn{1}{c}{out}&\multicolumn{1}{c}{spread}&\multicolumn{1}{c}{q0}&\multicolumn{1}{c}{q0.25}&\multicolumn{1}{c}{q0.5}&\multicolumn{1}{c}{q0.75}&\multicolumn{1}{c}{q1}&\multicolumn{1}{c}{summary}\tabularnewline
\midrule
\endhead
\midrule
\endfoot
\label{as.matrix}
\textbf{ Poa trivialis }&hl&yes&0.601&100&11& 75& 64&3&3.0&8.00& 8.0&18.00&18.0&V (3/8/18, n = 11)\tabularnewline
Trifolium repens&hl&&0.412& 64& 7& 93& 86&3&0.0&0.00& 4.0& 6.00& 8.0&IV (0/4/8, n = 7)\tabularnewline
Cerastium holosteoides&hl&&0.364& 73& 8&119&111&3&0.0&0.15& 0.3& 0.30& 0.7&IV (0/0.3/0.7, n = 8)\tabularnewline
\textbf{ Trisetum flavescens }&hl&yes&0.351&100&11&142&131&3&0.7&8.00&18.0&18.00&18.0&V (0.7/18/18, n = 11)\tabularnewline
\textbf{ Dactylis glomerata }&hl&&0.309&100&11&158&147&3&0.7&1.85& 4.0& 4.00&18.0&V (0.7/4/18, n = 11)\tabularnewline
Carex sylvatica&hl&yes&0.236& 45& 5& 72& 67&3&0.0&0.00& 0.0& 1.85& 3.0&III (0/0/3, n = 5)\tabularnewline
Trifolium pratense&hl&&0.236& 82& 9&139&130&3&0.0&0.70& 0.7& 3.50& 4.0&V (0/0.7/4, n = 9)\tabularnewline
Carex hirta&hl&yes&0.211& 45& 5& 10&  5&2&0.0&0.00& 0.0& 0.50& 4.0&III (0/0/4, n = 5)\tabularnewline
Ranunculus repens&hl&yes&0.209& 82& 9& 38& 29&3&0.0&0.30& 0.7& 1.85& 8.0&V (0/0.7/8, n = 9)\tabularnewline
Rumex obtusifolius&hl&yes&0.205& 36& 4& 17& 13&3&0.0&0.00& 0.0& 0.50& 4.0&II (0/0/4, n = 4)\tabularnewline
Festuca pratensis s.str.&hl&&0.203& 82& 9&111&102&3&0.0&2.35& 4.0& 6.00&38.0&V (0/4/38, n = 9)\tabularnewline
Phleum pratense&hl&&0.198& 45& 5& 37& 32&3&0.0&0.00& 0.0& 2.35& 8.0&III (0/0/8, n = 5)\tabularnewline
Veronica arvensis&hl&yes&0.195& 45& 5& 16& 11&3&0.0&0.00& 0.0& 0.30& 0.3&III (0/0/0.3, n = 5)\tabularnewline
Poa pratensis&hl&yes&0.190& 36& 4& 38& 34&3&0.0&0.00& 0.0& 2.35& 8.0&II (0/0/8, n = 4)\tabularnewline
Symphytum officinale&hl&&0.182& 18& 2&  7&  5&3&0.0&0.00& 0.0& 0.00& 0.7&I (0/0/0.7, n = 2)\tabularnewline
Cirsium oleraceum&hl&&0.176& 64& 7& 70& 63&3&0.0&0.00& 0.7& 0.70& 8.0&IV (0/0.7/8, n = 7)\tabularnewline
Ranunculus acris&hl&&0.176& 91&10&155&145&3&0.0&0.70& 0.7& 1.85& 4.0&V (0/0.7/4, n = 10)\tabularnewline
Carum carvi&hl&yes&0.175& 73& 8& 58& 50&3&0.0&0.15& 0.7& 0.70& 8.0&IV (0/0.7/8, n = 8)\tabularnewline
Plantago lanceolata&hl&&0.134& 55& 6&116&110&3&0.0&0.00& 0.3& 0.30& 4.0&III (0/0.3/4, n = 6)\tabularnewline
Myosotis sylvatica&hl&&0.130& 18& 2& 19& 17&3&0.0&0.00& 0.0& 0.00& 0.7&I (0/0/0.7, n = 2)\tabularnewline
Lathyrus pratensis&hl&&0.117& 27& 3& 66& 63&3&0.0&0.00& 0.0& 0.15& 0.3&II (0/0/0.3, n = 3)\tabularnewline
Glechoma hederacea&hl&yes&0.117& 27& 3& 12&  9&3&0.0&0.00& 0.0& 0.15& 0.3&II (0/0/0.3, n = 3)\tabularnewline
Stellaria graminea&hl&&0.117& 27& 3& 58& 55&3&0.0&0.00& 0.0& 0.15& 0.3&II (0/0/0.3, n = 3)\tabularnewline
Heracleum sphondylium&hl&&0.114& 27& 3& 54& 51&3&0.0&0.00& 0.0& 0.15& 4.0&II (0/0/4, n = 3)\tabularnewline
Crepis biennis&hl&&0.114& 45& 5& 80& 75&3&0.0&0.00& 0.0& 0.50& 8.0&III (0/0/8, n = 5)\tabularnewline
Elymus repens&hl&yes&0.111& 18& 2&  6&  4&2&0.0&0.00& 0.0& 0.00&18.0&I (0/0/18, n = 2)\tabularnewline
Mentha longifolia&hl&yes&0.107& 18& 2&  3&  1&2&0.0&0.00& 0.0& 0.00& 4.0&I (0/0/4, n = 2)\tabularnewline
Bellis perennis&hl&&0.098& 18& 2& 24& 22&3&0.0&0.00& 0.0& 0.00& 4.0&I (0/0/4, n = 2)\tabularnewline
Taraxacum officinale agg.&hl&&0.084& 64& 7& 74& 67&3&0.0&0.00& 0.3& 0.70& 0.7&IV (0/0.3/0.7, n = 7)\tabularnewline
Rumex acetosa&hl&&0.083& 73& 8&115&107&3&0.0&0.15& 0.3& 0.70& 4.0&IV (0/0.3/4, n = 8)\tabularnewline
Campanula patula&hl&&0.078& 18& 2& 61& 59&3&0.0&0.00& 0.0& 0.00& 0.3&I (0/0/0.3, n = 2)\tabularnewline
Aegopodium podagraria&hl&&0.073& 45& 5& 66& 61&3&0.0&0.00& 0.0& 0.70& 4.0&III (0/0/4, n = 5)\tabularnewline
Arrhenatherum elatius&hl&&0.061& 18& 2& 43& 41&3&0.0&0.00& 0.0& 0.00& 8.0&I (0/0/8, n = 2)\tabularnewline
Petasites hybridus&hl&&0.061& 18& 2& 12& 10&2&0.0&0.00& 0.0& 0.00& 8.0&I (0/0/8, n = 2)\tabularnewline
Veronica chamaedrys&hl&&0.059& 64& 7&102& 95&3&0.0&0.00& 0.3& 0.30& 3.0&IV (0/0.3/3, n = 7)\tabularnewline
Achillea millefolium agg.&hl&&0.058& 45& 5& 69& 64&3&0.0&0.00& 0.0& 0.30& 0.7&III (0/0/0.7, n = 5)\tabularnewline
Rhinanthus alectorolophus&hl&&0.052& 27& 3&101& 98&3&0.0&0.00& 0.0& 0.15& 4.0&II (0/0/4, n = 3)\tabularnewline
Vicia cracca&hl&&0.048& 36& 4& 92& 88&3&0.0&0.00& 0.0& 0.30& 0.7&II (0/0/0.7, n = 4)\tabularnewline
Galium album s.str.&hl&&0.045& 27& 3& 51& 48&3&0.0&0.00& 0.0& 0.15& 3.0&II (0/0/3, n = 3)\tabularnewline
Prunella vulgaris&hl&&0.045& 36& 4& 70& 66&3&0.0&0.00& 0.0& 0.30& 0.7&II (0/0/0.7, n = 4)\tabularnewline
Anthoxanthum odoratum&hl&&0.040& 36& 4&106&102&3&0.0&0.00& 0.0& 0.50& 4.0&II (0/0/4, n = 4)\tabularnewline
Deschampsia cespitosa&hl&&0.035& 18& 2& 11&  9&2&0.0&0.00& 0.0& 0.00& 4.0&I (0/0/4, n = 2)\tabularnewline
Alopecurus pratensis&hl&&0.034& 45& 5& 24& 19&3&0.0&0.00& 0.0& 0.70& 4.0&III (0/0/4, n = 5)\tabularnewline
Cynosurus cristatus&hl&&0.034& 55& 6& 96& 90&3&0.0&0.00& 0.7& 2.35& 4.0&III (0/0.7/4, n = 6)\tabularnewline
Carex pallescens&hl&&0.032& 18& 2& 73& 71&3&0.0&0.00& 0.0& 0.00& 0.7&I (0/0/0.7, n = 2)\tabularnewline
Leucanthemum ircutianum&hl&&0.032& 18& 2& 75& 73&3&0.0&0.00& 0.0& 0.00& 0.7&I (0/0/0.7, n = 2)\tabularnewline
Alchemilla vulgaris agg.&hl&&0.032& 73& 8&124&116&3&0.0&0.15& 0.3& 0.30& 0.7&IV (0/0.3/0.7, n = 8)\tabularnewline
Centaurea jacea&hl&&0.025& 27& 3& 95& 92&3&0.0&0.00& 0.0& 0.15& 4.0&II (0/0/4, n = 3)\tabularnewline
Cruciata laevipes&hl&&0.023& 18& 2& 32& 30&3&0.0&0.00& 0.0& 0.00& 0.7&I (0/0/0.7, n = 2)\tabularnewline
Festuca nigrescens&hl&&0.020& 27& 3&108&105&3&0.0&0.00& 0.0& 0.15& 4.0&II (0/0/4, n = 3)\tabularnewline
Vicia sepium&hl&&0.020& 27& 3& 50& 47&3&0.0&0.00& 0.0& 0.15& 0.3&II (0/0/0.3, n = 3)\tabularnewline
Colchicum autumnale&hl&&0.015& 27& 3& 98& 95&3&0.0&0.00& 0.0& 0.15& 0.7&II (0/0/0.7, n = 3)\tabularnewline
Avenula pubescens&hl&&0.015& 36& 4& 78& 74&3&0.0&0.00& 0.0& 0.70& 4.0&II (0/0/4, n = 4)\tabularnewline
Rhinanthus minor&hl&&0.014& 18& 2& 45& 43&3&0.0&0.00& 0.0& 0.00& 0.3&I (0/0/0.3, n = 2)\tabularnewline
Chaerophyllum aureum&hl&&0.013& 27& 3& 55& 52&3&0.0&0.00& 0.0& 0.35& 4.0&II (0/0/4, n = 3)\tabularnewline
Geranium phaeum&hl&&0.011& 18& 2&  7&  5&3&0.0&0.00& 0.0& 0.00& 0.7&I (0/0/0.7, n = 2)\tabularnewline
Chaerophyllum hirsutum&hl&&0.011& 27& 3& 64& 61&3&0.0&0.00& 0.0& 0.35& 0.7&II (0/0/0.7, n = 3)\tabularnewline
Hypericum maculatum s.str.&hl&&0.009& 27& 3& 97& 94&3&0.0&0.00& 0.0& 0.15& 0.7&II (0/0/0.7, n = 3)\tabularnewline
Agrostis capillaris&hl&&0.007& 18& 2& 61& 59&3&0.0&0.00& 0.0& 0.00& 0.7&I (0/0/0.7, n = 2)\tabularnewline
Senecio subalpinus&hl&&0.007& 18& 2& 10&  8&2&0.0&0.00& 0.0& 0.00& 0.3&I (0/0/0.3, n = 2)\tabularnewline
\midrule
Occuring only once& \multicolumn{14}{p{150mm}}{Alchemilla vulgaris s.str., Betonica officinalis, Brachypodium pinnatum, Bromus erectus, Campanula trachelium, Cardamine hirsuta, Cardaminopsis halleri, Carduus crispus, Carex flacca, Carex leporina, Cirsium arvense, Clinopodium vulgare, Filipendula vulgaris, Galium boreale, Galium verum, Geum urbanum, Holcus lanatus, Juncus articulatus, Leontodon autumnalis, Leontodon hispidus, Lilium bulbiferum, Lotus corniculatus, Lysimachia nummularia, Medicago lupulina, Myosotis arvensis, Myosotis nemorosa, Persicaria bistorta, Pimpinella major, Potentilla reptans, Ranunculus bulbosus, Ranunculus nemorosus, Stachys alpina, Stachys sylvatica, Tragopogon orientalis, Trollius europaeus, Urtica dioica, Valeriana officinalis, Veratrum album, Veronica serpyllifolia, Willemetia stipitata}\tabularnewline
\midrule
accuracy& \multicolumn{14}{p{150mm}}{.}\tabularnewline
altitude& \multicolumn{14}{p{150mm}}{348/756.5/\textbf{ 788 }/928/973}\tabularnewline
author& \multicolumn{14}{p{150mm}}{.}\tabularnewline
coverscale& \multicolumn{14}{p{150mm}}{.}\tabularnewline
e_coord& \multicolumn{14}{p{150mm}}{15.29739/15.375995/\textbf{ 15.39633 }/15.409275/15.63057}\tabularnewline
fels_antl& \multicolumn{14}{p{150mm}}{.}\tabularnewline
n_coord& \multicolumn{14}{p{150mm}}{47.76125/47.763705/\textbf{ 47.76925 }/47.87654/48.04266}\tabularnewline
nr_gl_ges& \multicolumn{14}{p{150mm}}{.}\tabularnewline
oevdat& \multicolumn{14}{p{150mm}}{.}\tabularnewline
releve_nr& \multicolumn{14}{p{150mm}}{14492/14553/\textbf{ 14636 }/14684/14719}\tabularnewline
schutt_ant& \multicolumn{14}{p{150mm}}{.}\tabularnewline
sp_count& \multicolumn{14}{p{150mm}}{.}\tabularnewline
surf_area& \multicolumn{14}{p{150mm}}{.}\tabularnewline
waypoint& \multicolumn{14}{p{150mm}}{524/534/\textbf{ 574 }/608/610}\tabularnewline
x_coord& \multicolumn{14}{p{150mm}}{15.297/15.376/\textbf{ 15.396 }/15.4095/15.631}\tabularnewline
y_coord& \multicolumn{14}{p{150mm}}{47.761/47.7635/\textbf{ 47.769 }/47.8765/48.043}\tabularnewline
bezirk& \multicolumn{14}{p{150mm}}{Bruck an der Mur: 7; Lilienfeld: 3; Scheibbs: 1}\tabularnewline
date& \multicolumn{14}{p{150mm}}{2014-06-11: 2; 2014-06-21: 2; 2014-06-25: 2; 2014-06-28: 2; 2014-06-04: 1; 2014-06-10: 1; 2014-06-24: 1}\tabularnewline
diag_ms& \multicolumn{14}{p{150mm}}{Poo-Trisetetum: 11}\tabularnewline
locality& \multicolumn{14}{p{150mm}}{Halltal: 7; Lassingbachtal: 1; Ötschergebiet: 1; Traisental: 1; Türnitztal: 1}\tabularnewline
moss_ident& \multicolumn{14}{p{150mm}}{Y: 11}\tabularnewline
observer& \multicolumn{14}{p{150mm}}{Staudinger, Markus: 11}\tabularnewline
ordnung& \multicolumn{14}{p{150mm}}{Arrhenatheretalia: 11}\tabularnewline
orig_diag& \multicolumn{14}{p{150mm}}{Poo-Trisetetum: 10; Filipendulo-Arrhenatheretum: 1}\tabularnewline
project& \multicolumn{14}{p{150mm}}{Kartierung Halltall,Walstern: 7; Bergmähwiesen NÖ: 4}\tabularnewline
quadrant& \multicolumn{14}{p{150mm}}{8258/1: 6; 7959/4: 1; 8058/3: 1; 8157/2: 1; 8158/1: 1; 8258/2: 1}\tabularnewline
region& \multicolumn{14}{p{150mm}}{Steirische Kalkvoralpen: 7; NÖ Kalkvoralpen: 4}\tabularnewline
remarks& \multicolumn{14}{p{150mm}}{Annarotte S Annaberg, WP 344: 1; E Traisen, WP 420; Pferdeweide; ; ; ; ; : 1; oberes Halltal, WP 518; typische Talbodenfettwiese; ; ; ; ; ; : 1; oberes Halltal, WP 524; Talbodenfettwiese; ; ; ; ; : 1; oberes Halltal, WP 534; Talbodenfettwiese; ; ; : 1; oberes Halltal, WP 571 Fettwiese: 1; oberes Halltal, WP 577: 1; oberes Halltal, WP 608 Talbodenfettwiese: 1; S Siebenbrunn, WP 412; Fläche stark beschattet mit Verbrachungszeigern; ; ; ; : 1; unteres Halltal, Thaleralm, WP 610 wahrscheinlich Einsaatwiese: 1; Wiese NE Gösing an der Mariazeller Bahn, WP 396; : 1}\tabularnewline
verband& \multicolumn{14}{p{150mm}}{Trisetion: 11}\tabularnewline
\bottomrule
\end{longtable}

\newpage

\end{document}
